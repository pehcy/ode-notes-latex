\chapter{Solving First-order Differential Equation}

\begin{theorem}
    If function $f(x)$ and function $g(x)$ are continuous, then equation 
    \[
        \int f(x) dx = \int g(y) dy + C
    \]
\end{theorem}

\begin{example}
    Solve $e^{x+y}\> dy - 1\> dx = 0$.
\end{example}
\begin{solution}
    The DE is separable and can be formulate as 
    \begin{align*}
        e^{x+y}\> dy = 1\> dx \quad &\Rightarrow e^x * e^y \> dy = 1\> dx\\
        &\Rightarrow e^y \> dy = e^{-x}\> dx
    \end{align*}
    
    Integrating both sides we have
    \begin{align*}
        \int e^y \> dy = \int e^{-x}\> dx \quad &\Rightarrow e^y = -e^{-x} + C & 
        e^y > 0 \text{ so that RHS }> 0 \\
        &\Rightarrow y = \ln |-e^{-x} + C| &\text{general solution in implicit form}
    \end{align*}
\end{solution}

\begin{example}
    Find all solutions to $y' = -2y^2x$. Be sure to describe any singular solutions if there is one.
\end{example}
\begin{solution}
    Is this DE separable? Yes, since it can be written as 
    \[
        - \frac{dy}{y^2} = 2x\> dx
    \]
    Integrating both sides of the equation, we have 
    \begin{align*}
        -\frac{1}{2y} = -\frac{1}{2}x^2 + c_1 &\Rightarrow \frac{1}{y} = x^2 - 2c_1\\
        &\Rightarrow y = \frac{1}{x^2 - 2c_1}
    \end{align*}

    By inspection, $y = 0$ is another solution (obvious solution).

    Therefore, the solutions are $y = 0$ and $y = (x^2 - 2c)^{-1} \quad \forall x \in \mathbb{R}$.
\end{solution}

\section{Exact Equation}

The equation 

\begin{equation}
    M(x, y)\> dx + N(x,y)\> dy = 0
\end{equation}

is exact if $\exists F(x,y)$ such that $M\> dx + N\> dy = dF$. In this case, the solution to the DE 
is given by $dF = 0$ or $F(x,y) = C$, $C$ is a constant.

\begin{definition}[Total differential]
    Let $F(x,y)$ be a function that has continuous first derivative in a 
    domain $D$.

    \begin{equation}
        dF = \frac{\partial F}{\partial x}dx + \frac{\partial F}{\partial y}dy 
        \quad \forall (x,y) \in D
    \end{equation}
\end{definition}

\begin{theorem}[Test for Exactness]
    Suppose $M,N, \frac{\partial M}{\partial y}$, and $\frac{\partial N}{\partial x}$ are continuous in the open 
    rectanger $R: a< x<b, \> c < y < d$. Then 

    \begin{equation}
        M(x,y)\> dx + N(x,y)\> dy = 0 \text{ if and only if } \frac{\partial M}{\partial y} = \frac{\partial N}{\partial x}
    \end{equation}
\end{theorem}
\begin{proof}
    ($\Rightarrow$) If $M(x,y)\> dx + N(x,y)\> dy = 0$ is exact, then we can find a potential 
    function $F$ such that $F_x = M$ and $F_y = N$. As the first-order partial derivatives of $M$ and 
    $N$ are continuous in $R$, according to the commutative law of partial derivative operator,

    \begin{equation}
        \frac{\partial M}{\partial y} = F_{xy} = F_{yx} = \frac{\partial N}{\partial x}
    \end{equation}

    at each point of $R$.

    ($\Leftarrow$) On the other hand, consider 
    
    \begin{equation}
        \frac{\partial M}{\partial y} = \frac{\partial N}{\partial x}
    \end{equation}

    to prove $M(x,y)\> dx + N(x,y)\> dy = 0$ is exact, we must show that we can construct a function $F$
    such that $F_x = M$ and $F_y = N$.

    Let $\phi$ to be a function such that $\frac{\partial \phi}{\partial x} = M$. Then 
    \begin{equation}
        \frac{\partial^2 \phi}{\partial y \partial x} = \frac{\partial M}{\partial y} = \frac{\partial N}{\partial x}
    \end{equation}
    so that 
    \begin{equation}
        \frac{\partial N}{\partial x} = \frac{\partial^2 \phi}{\phi x \phi y}
    \end{equation}

    Integrating both sides with respect to $x$, we get 
    \begin{equation}
        N = \frac{\partial \phi}{\partial y} + B'(y)
    \end{equation}
\end{proof}

\begin{example}
    Solve $3x(xy - 2)\> dx + (x^3 + 2y)\> dy = 0$.
\end{example}
\begin{solution}
    The DE is in the form of $M\> dx + N\> dy = 0$, where 
    \[
        \begin{cases}
            M = \displaystyle \frac{1}{t^2} + \frac{1}{y^2} \\[1em]
            N = \displaystyle \frac{at + 1}{y^3}
        \end{cases}
    \]

    In order to make DE to be exact, we must have $M_y = N_t \Rightarrow a = \ldots$.
\end{solution}

\begin{example}
    Solve the initial-value problem

    \[
        (2x\cos y + 3x^2 y)\> dx + (x^3 - x^2 \sin y - y)\> dy = 0, \quad y(0) = 2
    \]
\end{example}
\begin{solution}
    First, we have to determine whether or not the equation is exact. Here 

    \[
        M = 2x\cos y + 3x^2y, \quad N = x^3 - x^2\sin y - y
    \]
    \[
        M_y = -2x\sin y + 3x^2, \quad N_x = 3x^2 - 2x\sin y
    \]

    Since $M_y = N_x = 3x^2 - 2x\sin y \quad \forall (x, y) \in \mathbb{R}^2$, the DE is exact in every rectangular
    domain $D$. Next, we must find $F$ such that

    \[
        \left\{
        \begin{array}{cc}
         F_x = M = 2x\cos y + 3x^2y & \quad \refstepcounter{equation}(a)\\
         M_y = x^3 - x^2\sin y - y & \quad \refstepcounter{equation}(b)
        \end{array}
        \right.
        \]
    
        From $(a) \Rightarrow F = \int (2x\cos y + 3x^2y)\> dx = x^2\cos y + x^3y + g(y)$.\quad (c)\\
        where $g$ is a function of $y$.

        Again, $(c) \Rightarrow F_y = -x^2\sin y + x^3 + g'(y)$ \quad (d)

        Now comparing (b) and (d), 
        \[
            g'(y) = -y \xrightarrow[]{\text{Integrate with respect to } y} g(y) = -\frac{y^2}{2} + C \quad 
            \text{where } C \text{ is an arbitrary constant}
        \]

        Thus, we have potential function 
        \[
            F = x^2\cos y + x^3y - \frac{1}{2}y^2 + C
        \]

        Hence a 1-parameter family of solutions is $F(x,y) = 0$ or 
        $\displaystyle x^2\cos y + x^3y - \frac{1}{2}y^2 + C$.

        Finally, we can now use the initial condition $y(0) = 2$ to find $C$: Subtituting 
        $x = 0, y = 2$ into the above solution, we obtain

        \[
            F(0, 2) = 0 + 0 - 2 + C = 0 \Rightarrow C = 2
        \]

        Therefore, the solution to the IVP is $\displaystyle x^2\cos y + x^3y - \frac{1}{2}y^2 + 2$.
\end{solution}

\begin{example}
    Determine the constant $a$ so that the equation 
    \[
        \frac{1}{t^2} + \frac{1}{y^2} + \biggl(\frac{at+1}{3}\biggr) \frac{dy}{dt} = 0
    \]
    is exact, and then solve the resulting equation.
\end{example}

\begin{theorem}
    The general solution to an exact equation $M(x,y)\>dx + N(x,y)\>dy = 0$ is 
    defined implicitly by 
    \begin{equation}
        F(x,y) = C
    \end{equation}
    where $F$ is a potential function of the DE and $C$ is an 
    arbitrary constant.
\end{theorem}
\begin{remark}
    We can find $F$ by solving the system of equations 
    \begin{equation}
        \begin{cases}
            \displaystyle 
            \frac{\partial F}{\partial x} = M\\[1em]
            \displaystyle 
            \frac{\partial F}{\partial y} = N
        \end{cases}
    \end{equation}
\end{remark}

\begin{proof}
    If $M(x,y)\>dx + N(x,y)\>dy = 0$ is exact, then $\exists$ a potential
    function $F$ such that $M(x,y)\>dx + N(x,y)\>dy = dF$.

    This gives us $dF = 0$ so that $F(x,y) = C$, where $C$ is an arbitrary 
    constant.
\end{proof}

\section{Making an Equation Exact: Integrating Factors}

Sometimes it is impossible to transform a nonexact DE that into an exact 
equation by multiplying it by a function. The resulting DE can be resolved using the technique
 of the previous section. However, it is impossible for a solution to be lost or gained as 
a result of the multiplication.

\begin{definition}
    If $M(x,y)\>dx + N(x,y)\>dy = 0$ is not exact but 
    $I(x,y)M(x,y)\>dx + N(x,y)I(x,y)\>dy = 0$ is exact, then $I(x,y)$ is called 
    an integrating factor of the DE.
\end{definition}
\begin{remark}
    We may be able to determine $I(x,y)$ from the equation 
    \begin{equation}
        \frac{\partial}{\partial y}(IM) = \frac{\partial}{\partial y}(IN)
    \end{equation}
\end{remark}

\begin{example}
    Verify that $I(x,y) = x^{-1}$ is an integrating factor of 
    $(x+y)\> dx + x\ln x \> dy = 0$ on the interval $(0, \infty)$. Hence, find the 
    solution for this DE.
\end{example}
\begin{solution}
    \[
        (x+y)\> dx + x\ln x \> dy = 0 \quad \quad (1)
    \]
    First, show that Eq(1) is not exact. Suppose $M = x+y$, and $N = x\ln x$.
    Then 
    \[
        M_y = 1, \quad N_x = x\biggl(\frac{1}{x}\biggr) + \ln x = 1 + \ln x
    \]

    Because $M_y \neq N_x$, so Eq(1) is not exact.

    Next, multiplying Eq(1) by $\displaystyle I(x,y) = \frac{1}{x}$, we obtain 
    \[
        \frac{(x+y)}{x}\>dx + \ln x \> dy = 0 \quad \quad (2)
    \]

    Now we show that Eq(2) is exact. Let $\displaystyle \tilde{M} = 1 + \frac{1}{y}$, 
    and $\tilde{N} = \ln x$, then 

    \[
        \frac{\partial \tilde{M}}{\partial y} = \frac{\partial \tilde{N}}{\partial x} = \frac{1}{x}
    \]

    The general solution will be $F(x,y) = C$, we can find $F$ by solving 
    the system of equations 

    \[
        \left\{
        \begin{array}{cc}
         \displaystyle \frac{\partial F}{\partial x} = 1+\frac{y}{x} & \quad \refstepcounter{equation}(3)\\[1em]
         \displaystyle \frac{\partial F}{\partial y} = \ln x & \quad \refstepcounter{equation}(4)
        \end{array}
        \right.
    \]

    Integrating Eq(3) with respect to $x$ we have 
    \[
        \int (3) dx \quad \Rightarrow F = \int \biggl(1 + \frac{y}{x}\biggr)\> dx 
        = x + y\ln x + g(y) \quad \quad (5)
    \]
    where $g(y)$ is a function of $y$ alone.

    Differentiate Eq(5) with respect to $y$, 
    \[
        \frac{\partial F}{\partial y} = \ln |x| + g'(y) \quad \quad (6)
    \]

    Now comparing Eq(4) and Eq(6), we obtain $g'(y) = g(y) = 0$.
    
    Thus, the general solution is $F = x + y\ln |x| + 0$.

\end{solution}

\section*{Summary: Solving 1st order DE}

\begin{enumerate}
    \item Separable equation: $f(x) \>dx = g(y) \>dy$. (Method of solving: integrating both sides)
    \item Exact DE: Use \textbf{Exactness Test}, whether 
        $\displaystyle \frac{\partial M}{\partial y} = \frac{\partial N}{\partial x}$
\end{enumerate}