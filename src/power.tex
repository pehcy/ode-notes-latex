\chapter{Power Series Solutions}

\begin{example}
    Solve the IVP 
    \[
        \left\{
        \begin{array}{cc}
         y' + 2xy = x^3 & \quad \refstepcounter{equation}(1)\\[1em]
         y(1) = 1 & \quad \refstepcounter{equation}(2)
        \end{array}
        \right.
    \]
    Find the first 3 nonzero-terms of the Taylor series of $y(x)$ about $x=1$.
\end{example}
\begin{solution}
    The Taylor series of $y(x)$ about $x=1$ is 
    \[
        y(x) = \sum^\infty_{n=0} \frac{y^{(n)}(1)}{n!}(x-1)^n = 
        y(1) + y'(1)(x-1) + \frac{y''(1)}{2!}(x-1)^2 + 
        \frac{y^{(3)}(1)}{3!}(x-1)^2 + \cdots
    \]
    by definition.

    It remains to find $y^{(n)}(1)$ for $n \geq 1$ (until we get the first 3 nonzero-term)

    From Eq(1), 
    \[
        y'(x) = x^3 -2xy(x) \Rightarrow y'(1) = 1^3 - 2(1)y(1) = 1 - 2 = -1 
    \]

    From Eq(1), differentiate again,
    \[
        y''(x) = 3x^2 - \biggl(2x\frac{dy}{dx} + y(2)\biggr) = 3x^2 - 2xy' -2y \quad \quad (3)
    \]

    Substitute $y(1)= 1, y'(1) = -1$ into Eq(3),
    \[
        y''(1) = 3(1)^2 - 2(1)(-1) - 2 = 3
    \]

    Thus the Taylor series is 
    \begin{align*}
        y(x) &= 1 + (-1)(x-1) + \frac{3}{2!}(x-1)^2 + \cdots \\
        &= 1 + (1-x) + \frac{3}{2!}(x-1)^2 + \cdots \\ 
        &= 2 - x + \frac{3}{2!}(x-1)^2 + \cdots
    \end{align*}
\end{solution}

Q: Here is the question, under what condition does a DE has a solution of the form 

\begin{equation}
    \sum^\infty_{n=0} a_n (x - x_0)^n 
\end{equation}

This bring us to another section: which is about analytic at a point for a series.

\newpage
\section{Analytic at a point}

\begin{definition}
    If the Taylor series of $f$, where 
    \begin{equation}
        f(x) = \sum^\infty_{n=0} \frac{f^{(n)}(a)}{n!}(x-a)^n
    \end{equation}

    exists and converges to $f(x) \quad \forall x \in I$, an open interval containing $x = a$, 
    then function $f$ is analytic at $x = a$. 

\end{definition}

\subsection*{Example of Analytic functions}
All polynomials $P(x) = a_0 + a_1x + a_2x^2 + \cdots$ are analytic $\forall x \in \mathbb{R}$.

\begin{example}
    \textbf{Legendre Equation}

    \[
        (1 - x^2)y'' + 2xy' + \lambda y = 0 \quad \quad (1)
    \]

    Find a power series solution for this DE.
\end{example}
\begin{solution}
    In the standard form $y'' + p(x)y' + q(x)y = 0$, we have 
    \[
        p(x) = \frac{2}{1 - x^2}, \quad q(x) = \frac{\lambda}{1 - x^2}
    \]
    Both $p$ and $q$ are analytic at $x=0$. As $x=0$ is an ordinary point
    , Eq(1) will have two linearly independent solutions of the form 
    \[
        \sum^\infty_{n=0}a_nx^n 
    \]

    Substitute $\displaystyle y = \sum^\infty_{n=0} a_nx^n, y' = \sum^\infty_{n=1}a_nx^{n-1}$,
    $\displaystyle y'' = \sum^\infty_{n=2} n (n-1)a_nx^{n-2} \quad \quad (2)$

    and,

    \begin{align*}
        -x^2y'' = \sum^\infty_{n=2} -(n-1)na_nx^{n} &\Rightarrow 2xy' = \sum^\infty_{n=1} 2na_nx^n = 2a_1x + \sum^\infty_{n=2} 2na_nx^n\\
        &\Rightarrow \lambda y = \sum^\infty_{n=0} \lambda a_n x^n = \lambda a_0 + \lambda a_1 x + \sum^\infty_{n=2} \lambda a_n x^n
    \end{align*}

    into Eq(1).

    Replacing $n$ to $n+2$ in Eq(2), we obtain

    \begin{align*}
        (2) &= \sum^\infty_{n+2=2} (n+2)(n+1) a_{n+2}\>  x^{n+2-2} \\ 
        &= \sum_{n=0}^{\infty} (n+1)(n+2)a_{n+2}\> x^n \\
        &= (0+2)(0+1)a_2 + (1+2)(1+1)a_3 + \sum^\infty_{n=2}(n+1)(n+2)a_{n+2}x^n
    \end{align*}

    So that 

    \[
        (2a_2 + \lambda a_0) + (6a_3 + 2a_1 + \lambda a_1) x + \sum_{n=2}^{\infty} \biggl\{(n+1)(n+2)a_{n+2}
        - n(n-1)a_n + 2na_n + \lambda a_n \biggr\} x^n = 0    
    \]

    Equating coefficients both sides to zero,

    \[
        \left\{
        \begin{array}{cc}
         2a_2 + \lambda q = 0 & \quad \refstepcounter{equation}(3)\\[0.8em]
         6a_3 + (2 + \lambda) a_1 = 0 & \quad \refstepcounter{equation}(4)
        \end{array}
        \right.
    \]

    \subsection*{[Recurrence relation]}
    \[
        (n+1)(n+2) a_{n+2} + [2n + \lambda - n(n-1)]a_n = 0 \text{ for } n \geq 2
    \]

    Write all the $a_n$'s in terms of $a_0$ and $a_1$,

    \[
        (3) \Rightarrow a_2 = -\frac{\lambda}{2} a_0
    \]

    \[
        (4) \Rightarrow a_3 = - \frac{(2 + \lambda)}{6} a_1
    \]

    \begin{align*}
        (5) \Rightarrow a_4 &= \frac{2(2-3)-\lambda}{4 (3)}\> a_0\\
        &= \frac{-2-\lambda}{4(3)} \biggl(\frac{\lambda}{2}\biggr)\> a_0\\
        &= \frac{\lambda (\lambda + 2) a_0}{4!}
    \end{align*}

    The solution is 

    \begin{align*}
        y &= (a_0 + a_2x^2 + a_4x^4 + \cdots) + (a_1 + a_3x^3 + a_5x^5 + \cdots)\\
        &= a_0 \biggl(1 - \frac{\lambda}{2}x^2 + \frac{\lambda (\lambda+2)}{4!}x^4 + \cdots\biggr) + 
        a_1 \biggl(x - \frac{\lambda + 2}{3!}x^3 + \frac{\lambda (\lambda+2)}{5!}x^5 + \cdots\biggr)
    \end{align*}

\end{solution}

\section{Regular Singular Point}

\begin{theorem}[Frobenius Theorem]
    Given an equation $P(x)y'' + Q(x)y' + R(x)y = 0$, if $x$ is a regular singular point at $\mathbb{R}$, 
    then there exists a solution of the form 

    \[
        y = (x - x_0)^n \sum^\infty_{n=0} a_n (x-x_0)^n = \sum^\infty_{n=0} a_n (x-x_0)^{2n}
    \]
\end{theorem}

\begin{example}
    Solve the DE
    \[
        2xy'' + y' - y = 0
    \]
    and check whether $x = 0$ is a regular singular point of of this equation.
\end{example}
\begin{solution}
    The DE is 
    \[
        2xy'' + y' - y = 0 \quad \quad (1)
    \]

    Firstly, we need to check whether $x = 0$ is a RSP of Eq(1).

    By Theorem 3.1,  (1) has a Frobenius solution of the form $\displaystyle \sum^\infty_{n=0} a_n x^{n+r}$.

    To find the solutions, substitute 

    \[
        y = \sum^\infty_{n=0} a_n x^{n+r}
    \]

    \[
        y' = \sum^\infty_{n=0} (n+r) a_n x^{n+r-1} = ra_0 x^{r-1} + \sum^\infty_{n=1} (n+r)a_nx^{n+r-1}
    \]

    \[
        y'' = \sum^\infty_{n=0} (n+r)(n+r-1) a_n x^{n+r-2}
    \]

    into Eq(1),

    \begin{align*}
        &2xy'' = \sum^\infty_{n=0} 2(n+r)(n+r-1)a_n x^{n+r-1}\\
        &\Rightarrow 2xy'' = 2r(r-1)a_0x^{r-1} + \sum^\infty_{n=1} 2(n+r)(n+r-1)x^{n+r-1} a_n
    \end{align*}

    \textbf{Assumption:} Assume that $a_0 \neq 0$, 
    \begin{align*}
        (2) &\Rightarrow 2r(r-1) + r = 0\\
        &\Rightarrow 2r^2 - r = 0 & \text{Indicial equation}\\
        &\Rightarrow r = 0 \quad \text{ or } \quad r = \frac{1}{2} & \text{Indicial roots}
    \end{align*}

    Case 1: When $r = 0$, from Eq(3) we have 

    \[
        (3) \Rightarrow [2(n+1)n + (n+1)]a_{n+1} = a_n \quad \Rightarrow \quad a_{n+1} = \frac{a_n}{(n+1)(2n+1)}\> \forall 
        n \geq 0
    \]

    Iterate through $n = 0, 1, 2, \ldots$ and find $a_n$ in terms of $a_0$,

    \begin{align*}
        &n = 0: &a_1 = \frac{a_0}{1 \times 1} = a_0\\
        &n = 1: &a_2 = \frac{a_1}{2 \times 3} = \frac{a_0}{(1 \times 2)(1 \times 3)}\\
        &n = 2: &a_3 = \frac{a_2}{3 \times 5} = \frac{a_0}{(1 \times 2 \times 3)(1\times 3 \times 5)}\\
        &n = 3: &a_4 = \frac{a_3}{4 \times 7}
        = \frac{a_0(2 \times 4 \times 6 \times 8)}{(1 \times 2 \times 3 \times 4)(1 \times 3 \times 5 \times 7)(2 \times 4 \times 6 \times 8)}\\
        & &= \frac{a_0}{(1 \times 2 \times 3 \times 4)(1 \times 3 \times 5 \times 7)}
    \end{align*}

    by mathematical induction, $y_1$ can be express as 
    \[
        y_1 = a_0 \biggl[1 + \sum_{n=1}^{\infty} \frac{x^n}{n! \> \prod_{i=1}^{n} (2n-1) }\biggr],\quad  n \in \mathbb{Z}^+
    \]


    Case 2: When $r = 1/2$, from Eq(3) we have

    \begin{align*}
        &n = 0: &a_1 = \frac{a_0}{1 \times 3} = \frac{1}{3}a_0\\
        &n = 1: &a_2 = \frac{a_1}{5 \times 2} = \frac{a_0}{(1 \times 3 \times 5)(1 \times 2)}\\
        &n = 2: &a_3 = \frac{a_2}{7 \times 3} = \frac{a_0}{(1 \times 3 \times 5 \times 7)(1\times 2 \times 3)}\\
        &n = 3: &a_4 
        = \frac{a_0}{(1 \times 3 \times 5 \times 7 \times 9)(1 \times 2 \times 3 \times 4)}\\
        &\vdots &\vdots 
    \end{align*}

    The second solution is 
    \begin{align*}
        y_2 = \sum^\infty_{n=0} a_n x^{n+1/2} &= x^{1/2} \sum^\infty_{n=0} a_n x^n\\
        &= x^{1/2} [a_0 + a_1 + a_2 + \cdots]\\
        &= x^{1/2} \biggl[1 + \frac{x}{1 \times 3} + \frac{x^2}{(1\times 3 \times 5)(1 \times 2) 
        } + \frac{x^3}{(1 \times 3 \times 5 \times 7) (1 \times 2 \times 3)} + \cdots\biggr]
    \end{align*}
    which can be written as 
    \[
        y_2 = x^{1/2} \biggl[1 + \sum_{n=1}^{\infty} \frac{x^n}{n! \> \prod_{i=1}^{n} (2n+1) }\biggr]
    \]

    By inspection, $y_1$ and $y_2$ are not scalar multiples, implies that they are linearly independent.
    Therefore the general solution of Eq(1) is $y = c_1y_1 + c_2y_2$, where $c_1$ and $c_2$ are 
    arbitrary constants.
\end{solution}

\begin{remark}
    In general, we may not get two linearly independent solutions. We are guaranteed by 
    Frobenius theorem that thre is at least one solution in the form of 

    \begin{equation}
        y = \sum_{n=0}^{\infty} a_n (x-x_0)^n
    \end{equation}

    then we can use reduction of order to find for the 2nd linearly independent solution.
\end{remark}

\begin{example}
    For the given DE:
    \[
        xy'' + 3y' - y = 0
    \]
    Given that one of its solution is 
    \[
        y_1 = \sum_{n=0}^{\infty} \frac{2}{n!\, (n+2)!}\, x^n = 1 + \frac{1}{3}x + \frac{1}{24}x^2 + \frac{1}{360}x^3 + \cdots
    \]
    Find the 2nd LI solution using reduction of order.
\end{example}
\begin{solution}
    We write the original DE in standard form: $y" + p(x)y' + q(x)y = 0$, with $\displaystyle p(x) = \frac{3}{x}$.

    \begin{align*}
        \exp\big[{-\int p(x) \>dx}\big] &= \exp \big[{-\int -\frac{3}{x} \>dx}\big]\\
        &= e^{-3 |x|}\\
        &= \frac{1}{x^3}
    \end{align*}

    \begin{align*}
        \frac{e^{-\int p(x) \> dx}}{y_1^2} 
        = \frac{1}{x^3} \cdot \frac{1}{\biggl(1 + \frac{1}{3}x + \frac{1}{24}x^2 + \frac{1}{360}x^3 + \cdots\biggr)^2} \label{eq:1} \tag{a}
    \end{align*}

    To expand the squared series, we can apply the rule of multiplication for power series:

    \begin{mdframed}
        \subsubsection*{How to multiply two power series?}

        If there are two power series such that $f(x) = \sum_{n=0}^{N}a_n (x-x_0)^n$, $g(x) = \sum_{n=0}^{N}b_n (x-x_0)^n$, then 
        the multiplication of these two series are 

        \[
            f(x)g(x) = \sum_{n=0}^{N} c_n (x-x_0)^n
        \]

        where $c_n = a_0b_N + a_1b_{N-1} + a_2b_{N-2} + \cdots + a_Nb_0$.
    \end{mdframed}
    

    In this case, $f(x) = g(x)$ with $a_0 = b+0 = 1$, $a_1 = b_1 = \frac{1}{3}$, $a_2 = b_2 = \frac{1}{24}$. 

    \begin{align*}
        &c_0 = a_0b_0 = 1\\
        &c_1 = a_0b_1 + a_1b_0 = 1 \times \frac{1}{3} + \frac{1}{3} \times 1 = \frac{2}{3}\\
        &c_2 = a_0b_2 + a_1b_1 + a_2b_0 = \frac{1}{24} + \frac{1}{9} + \frac{1}{24} = \frac{7}{36}\\
        &c_3 = a_0b_3 + a_1b_2 + a_2b_1 + a_3b_0 = \frac{1}{360} + \frac{1}{72} + \frac{1}{72} = \frac{1}{360}
        = \frac{1}{30}\\
        &\vdots
    \end{align*}

    now we can continue to work on Eq\eqref{eq:1}.

    \begin{align*}
        \frac{e^{-\int p(x) \> dx}}{y_1^2} 
        &= \frac{1}{x^3} \cdot \frac{1}{\biggl(1 + \frac{1}{3}x + \frac{1}{24}x^2 + \frac{1}{360}x^3 + \cdots\biggr)^2}\\
        &= \frac{1}{x^3(c_0 + c_1\,x + c_2\,x^2 + c_3\,x^3 + \cdots)}\\
        &= \frac{1}{x^3 \biggl(1 + \frac{2}{3}x + \frac{7}{36}x^2 + \frac{1}{30}x^3 + \cdots\biggr)} \label{eq:2} \tag{b}
    \end{align*}

    Using long division to expand Eq\eqref{eq:2}:
    $$
    \begin{array}{rl}
        &\underline{\phantom{)} 1 - \frac{2}{3}x + \frac{1}{4}x^2 - \frac{19}{270}x^3 + \cdots}\\
        1+\frac{2}{3}x + \frac{7}{36}x^3 + \cdots & \smash{)}1 \\[0.6em]
        &\underline{\phantom{)} 1+\frac{2}{3}x + \frac{7}{36}x^3 + \cdots}\\[0.6em]
        &\phantom{1+}-\frac{2}{3}x - \frac{7}{36}x^2 - \frac{1}{30}x^3 + \cdots\\[0.6em]
        &\phantom{1+}-\underline{\frac{2}{3}x - \frac{4}{9}x^2 - \frac{7}{54}x^3 + \cdots}\\[0.6em]
        &\phantom{1+-\frac{2}{3}x}\frac{1}{4}x^2 + \frac{13}{136}x^3 + \cdots\\[0.6em]
        &\phantom{1+-\frac{2}{3}x}\underline{\frac{1}{4}x^2 + \frac{1}{6}x^3 + \cdots}\\[0.6em]
        &\phantom{1+-\frac{2}{3}x \frac{1}{4}x^2}-\frac{19}{270}x^3 + \cdots
    \end{array}$$

    we obtained 
    \begin{align*}
        \frac{e^{-\int p(x) \> dx}}{y_1^2}
        &= \frac{1}{x^3} \biggl(1 - \frac{2}{3}x + \frac{1}{4}x^2 - \frac{19}{270}x^3 + \cdots\biggr)\\
        &= x^{-3} - \frac{2}{3}x^{-2} + \frac{1}{4}x^{-1} - \frac{19}{270} + \cdots
    \end{align*}

    Now continue and integrate the result,

    \begin{align*}
        \int \frac{e^{-\int p(x) \> dx}}{y_1^2} \>dx &= \int \biggl(x^{-3} - \frac{2}{3}x^{-2} + \frac{1}{4}x^{-1} - \frac{19}{270} + \cdots \biggr)\> dx\\
        &= -\frac{1}{2}x^{-2} + \frac{2}{3}x^{-1} + \frac{1}{4} \ln |x| - \frac{19}{270}x + \cdots
    \end{align*}

    Finally, we can now apply the rule of reduction of order and find $y_2$.

    \begin{align*}
        y_2 &= y_1 \int \frac{e^{-\int p(x) \> dx}}{y_1^2} \>dx\\
        &= \biggl(1 + \frac{1}{3}x + \frac{1}{24}x^2 + \frac{1}{360}x^2 + \cdots \biggr) \biggl(-\frac{1}{2}x^{-2} + \frac{2}{3}x^{-1} + \frac{1}{4} \ln |x| - \frac{19}{270}x + \cdots\biggr)\\
        &= -\frac{1}{2}x^{-2} + \frac{2}{3}x^{-1} + \frac{1}{4} \ln |x| - \frac{19}{270}x + \cdots - \frac{1}{6}x^{-1} + \frac{1}{12}x \ln |x| - \frac{19}{810}x^2 + \cdots \\
        &\quad - \frac{1}{48} + \frac{1}{36}x + \frac{1}{96}x^2 \ln|x| + \frac{1}{540}x^2 + \cdots\\
    \end{align*}

\end{solution}

\begin{example}
    \begin{enumerate}
        \item Show that
        \[
            \sum^\infty_{k=0} a_{k+1}x^k + \sum^\infty_{k=0} a_kx^{k+1} = a_1 + \sum^\infty_{k=1} (a_{k-1} +
            a_{k+1})x^k
        \]
    \end{enumerate}
\end{example}
\begin{solution}
    From LHS,
    \begin{align*}
        \text{LHS} = \sum^\infty_{k=0} a_{k+1}x^k + \sum^\infty_{k=0} a_kx^{k+1}
        = \biggl( a_1 + \sum_{k=1}^{\infty} a_{k+1} x^k \biggr) + \sum_{k=0}^{\infty} a_k x^{k+1}
    \end{align*}
\end{solution}