\chapter{Second-order differential equations}
A second-order linear differential equation is an equation of the form

\begin{equation}
    a(x)y'' + b(x)y' + c(x)y = g(x)
\end{equation}

where $a(x), b(x), c(x)$, and $g(x)$ are functions of $x$. This DE is called \textbf{homogeneous} if 
$g(x) = 0$ for all $x$. Otherwise, it is called \textbf{non-homogeneous}.


We now consider what constitutes the so-called general solution of a homogeneous linear DE. To understand this, 
we first introduce the concepts of linear dependence and linear independent.

\begin{example}
    Find the longest interval in which the solution of the IVP 
    \[
        (x^2 - 3x)y'' + xy' - (x+3)y = 0, \quad y(1) = 2, \quad y'(1) = 1
    \]
    is certian to exist.
\end{example}
\begin{solution}
    By dividing by $(x^2 - 3x)$ we get the standard form 
    \[
        y'' + p(x)y' + q(x)y = r(x)
    \]
    where 
    \[
        p(x) = \frac{x}{x^2 - 3x}, \quad q(x) = -\, \frac{x+3}{x^2 - 3x}, \quad r(x) = 0
    \]
    being rational functions, both $p$, $q$, and $r$ are continuous everywhere except 
    at points $x = 0$ and $x = 3$. 

    Therefore the longest open interval, containing the initial point $x = 1$, in which all 
    the functions are continuous is $0 < x < 3$. Thus, $(0, 3)$ is the longest interval 
    which the theorem guarantees that the solution exists.
\end{solution}


\section{Superposition Principle}


\section{Linear dependence and linear independence}

\begin{definition}[Linear independence]
    The functions $f_1, f_2, \ldots, f_n$ are said to be linearly independent on an interval $I$ if there 
    exist constants $c_1, c_2, \ldots, c_n$ such that 

    \begin{equation}
        c_1 f_1(x) + c_2 f_2(x) + \cdots + c_n f_n(x) = 0
    \end{equation}
\end{definition}

Consider 
\begin{equation}
    S = \{ y : y'' + p(x)y' + q(x)y = 0 \}
\end{equation}

is a vector space with dimension 2. If $y_1$ and $y_2$ are two linearly 
independent solution to the HLDE, then its general solution is 

\begin{equation}
    y(x) = c_1 y_1(x) + c_2 y_2(x)
\end{equation}

where $c_1$ and $c_2$ are constants. The set of solution $\{y_1, y_2 \}$ is 
called the fundamental set of  solutions (a basis of $S$) to the HLDE.

To show that $\{y_1, y_2 \}$ is a fundamental set of solutions of HLDE. We can perform the Wronskian test, 
by Wronskian test we must show they are both \textbf{solutions} and \textbf{linearly independent}.

\begin{example}
    Determine whether the functions 
    \[
        f_1(x) = \sqrt{x}, \quad f_2(x) = \sqrt{x} + 8, \quad f_3 = 2, \quad f_4(x) = x^2 
    \]
    are linearly dependent on the interval $(0, \infty)$.

\end{example}
\begin{solution}
    By inspection, we have 
    \[
        f_2(x) = 1 \cdot f_1(x) + 4 \cdot 
    \]
\end{solution}

\begin{definition}[Wronskian]
    The \textbf{Wronskian} of two differential functions, said $f(x)$ and $g(x)$ 
    is the determinant

    \begin{equation}
        W(f(x), g(x)) = \begin{vmatrix}
            f(x) & g(x)\\ f'(x) & g'(x) 
        \end{vmatrix} = f(x)\,g'(x) - g(x)\, f'(x)
    \end{equation}
\end{definition}

\begin{theorem}[Wronskian Test for linearly dependence]
    Let $y_1(x)$ and $y_2(x)$ be two solutions of a second order homogeneous linear 
    differential equation 
    \begin{equation}
        y'' + p(x)y' q(x)y = 0
    \end{equation}
    where the coefficients $p(x)$ and $q(x)$ are continuous on an open interval $I$. 
    Then $y_1(x)$ and $y_2(x)$ are linearly dependent on $I$ and only if 
    
    \begin{equation}
        W(y_1(x), y_2(x)) = 0
    \end{equation}

    for every $x \in I$.
\end{theorem}
\begin{proof}
    $(\Rightarrow)$ Suppose $y_1(x)$ and $y_2(x)$ are linearly dependent on $I$. Then $y_1 = Cy_2$ for some 
    constant $C$. Hence, $y'_1 = Cy'_2$ and 
    
    \begin{equation*}
        W(y_1(x), y_2(x)) = \begin{vmatrix}
            y_1(x) & y_2(x)\\ y'_1(x) & y'_2(x) 
        \end{vmatrix} = \begin{vmatrix}
            Cy_1 & y_2(x)\\ Cy'_1(x) & y'_2(x) 
        \end{vmatrix} = 0
    \end{equation*}

    because the two columns are scalar multiples of each other.

    $(\Leftarrow)$ Suppose $W(y_1(x), y_2(x)) = 0$ for every $x \in I$. We must show that 
     $y_1$ and $y_2$ are scalar multiples of each other. There are two possible cases:

     \subsubsection*{Case 1}
     If $y_1(x) = 0$ for all $x \in I$, then $y_1 = 0 \cdot y_2$ and we are done here.

     \subsubsection*{Case 2}
     If $y_1(x) \neq 0$ for some $x \in I$. As $y_1$ is continuous on $I$, we must have a subinterval 
     $\mathfrak{I} \subseteq I$ such that 
     \[
        y_1(x) \neq 0 \quad \text{for all } x \in \mathfrak{I}
     \]
    
     Dividing 

     \[
        W(y_1(x), y_2(x)) = y_1y'_2 - y'_1y_2 = 0
     \]

     by $y_1^2$, we obtain 

     \[
        \frac{y_1y'_2 - y'_1y_2}{y_1^2} = \biggl( \frac{y_2}{y_1} \biggr)' = 0 \quad 
        \text{ or } \frac{y_2}{y_1} = C
     \]

     for some constant $C$. 

     Hence, $y_2(x) = Cy_1(x)$ for all $x \in \mathfrak{I}$.

     Since $\mathfrak{I} \subseteq I$, we may now apply the existence and uniqueness theorem to conclude that 

     \[
        y_2(x) = Cy_1(x)
     \]

     for all $x \in \mathfrak{I}$.
\end{proof}

Equivalently, we may induced the following theorem.

\begin{theorem}[Wronskian Test for linearly independence]
    The two solutions $y_1(x)$ and $y_2(x)$ of a second order homogeneous linear 
    differential equation
    \begin{equation}
        y'' + p(x)y' + q(x)y = 0
    \end{equation}
    are linearly independent on $I$ if and only if 
    \begin{equation}
        W(y_1(x), y_2(x)) \neq 0
    \end{equation}
    for every $x \in I$. Where $p$ and $q$ are continuous on an open interval $I$.
\end{theorem}

\begin{proof}
    let
    \begin{align*}
        &P(x): \text{The two solutions of the DE, } y_1(x) \text{ and } y_2(x) \text{ are linearly dependent.}\\
        &Q(x): W(y_1(x), y_2(x)) = 0
    \end{align*}

    Take the contraposition of the previous theorem,
    \begin{equation}
        \forall x \in I \> P(x) \leftrightarrow Q(x) \Longleftrightarrow \fbox{$\forall x \in I \> \lnot P(x) \leftrightarrow \lnot Q(x)$}
    \end{equation}

    which construct the theorem of \textbf{Wronskian Test for linearly independence}.
\end{proof}


\begin{remark}
    If Wronskian is nonzero, then the solutions $y_1$ and $y_2$ are linearly independent. 
    Otherwise, they are linearly dependent.
\end{remark}

\begin{example}
    Given the equation 

    \begin{equation}
        y'' - 4y = 0 \label{eqn:so-2}
    \end{equation}

    \begin{enumerate}
        \item Show that 
            \[
                y_1 = e^{2x}, \quad y_2 = e^{-2x}
            \]
            form a fundamental set of solutions of \eqref{eqn:so-2} on $\mathbb{R}$.
    \end{enumerate}
\end{example}

\section{Second-order HLDE with constant coefficients}

\begin{definition}
    A homogeneous 2nd order linear DE with constant coefficients has the form 

    \begin{equation}
        ay'' + by' + cy = 0 \label{eqn:so-1}
    \end{equation}

    where $a \neq 0$, $b$ and $c$ are constants.
\end{definition}

$ar^2 + br + c = 0$ or $P(r) = 0$ is called the characteristic equation (or auxiliary equation) 
The characteristic equation is always quadratic, and its two roots are

\begin{equation}
    r_1 = \frac{-b +\sqrt{b^2-4ac}}{2a}, \quad r_2 = \frac{-b -\sqrt{b^2-4ac}}{2a}
\end{equation}

Similar to what we had learned in quadratic equation, there are \textbf{three} possible forms 
for the general solution of \eqref{eqn:so-1} depending on the nature of the characteristic roots 
$r_1$ and $r_2$.

\begin{example}
    Find a general solution to
    \begin{enumerate}
        \item $2y'' - 2y' - 5y = 0$
        \item $y'' + 8y' + 16y = 0$
        \item $y'' + 2y' + 4y = 0$
    \end{enumerate}
\end{example}
\begin{solution}
    \begin{enumerate}
        The DEs above can be solve using the method of characteristic equation. 

        \item The roots of $2r^2 - 2r - 5 = 0$ are 
            \[
                r = \frac{-(-2) \pm \sqrt{(-2)^2 - 4(2)(-15)}}{2(2)} = \fbox{$\displaystyle \frac{1 \pm \sqrt{11}}{2}$}
            \]
            The two roots are real and distinct. Thus the general solution is 

            \[
                y = c_1 \exp \biggl[ \frac{(1 + \sqrt{11})}{2} \,x \biggr] + c_2 \exp \biggl[ \frac{(1 - \sqrt{11})}{2} \,x\biggr]
            \]

            where $c_1$ and $c_2$ are arbitrary constants.

        \item The characteristic equation is 
            \[
                r^2 + 8r + 16 = (r + 4)^2 = 0
            \]

            On solving, it has root $r = -4$ with multiplicity 2.
            Thus the general solution is 

            \[
                y = c_1e^{-4x} + c_2 \,xe^{-4x}
            \]

            where $c_1$ and $c_2$ are arbitrary constants.

        \item The characteristic equation $r^2 + 2r + 4 = 0$ has complex roots 
            \[
                r = \frac{-2 \pm \sqrt{2^2 - 4(1)(4)}}{2(1)} = \fbox{$-1 \pm \sqrt{3} i$}\>, \quad \text{with } i = \sqrt{-1}
            \]

            Thus, the general solution is 
            \[
                y = e^{-x}[c_1 \cos(\sqrt{3}x) + c_2 \sin(\sqrt{3}x)]
            \]

            where $c_1$ and $c_2$ are arbitrary constants.
    \end{enumerate}
\end{solution}


\subsection{Non-homogeneous DE with constant coefficients}

The general solution of non-homogeneous DE

\begin{equation}
    ay'' + by' + cy = g(x)
\end{equation}
is 
\begin{equation}
    y = y_h + y_p
\end{equation}
where $y_p$ is the general solution of the associated homogeneous equation 
\begin{equation}
    ay'' + by' + cy = 0
\end{equation}

\section{Method of undetermined coefficients}

\begin{example}
    Solve the initial value problem
    \[
        y'' + 4y = 12 \cos 2x, \quad y(0) = 3,\quad y'(0) = 4
    \]
\end{example}

\begin{solution}
    Consider the DE 

    \begin{equation*}
        y'' + 4y = 12 \cos 2x   \label{eq:sod2} \tag{{\color{red} $\bigstar$}}
    \end{equation*}

    This equation is a second-order non-homogeneous DE. The general solution should be 
    \[
        y = y_p + y_h
    \]

    We first finding the complementary solution $y_h$ using the method of 
    characteristic equation

    \begin{equation*}
        r^2 + 4 = 0 \Rightarrow r = \pm 2i
    \end{equation*}

    the general solution for $y_h$ is 

    \begin{equation*}
        y_h = c_1 \cos 2x + c_2 \sin 2x
    \end{equation*}

    where $c_1$ and $c_2$ are arbitrary coefficients.

    Find $y_p$, we consider 

    \[
        y_p = x[A \cos 2x + B \sin 2x]
    \]

    when differentiating we have 
    \begin{align*}
        y'_p &= A \cos 2x + B \sin 2x + x(-2A \cos 2x -2B \sin 2x]) \\[0.5em]
        y''_p &= (-2A \sin 2x + 2B \cos 2x) + (-2A \sin 2x + 2B \cos 2x)
        + x(-4A \cos 2x - 4B \sin 2x)
    \end{align*}

    which $A$ and $B$ are the constants that we need to solve.
    Now substituting $y_p$, $y'_p$, and $y''_p$ back into \eqref{eq:sod2}, the equation is 

    \[
        [-4A \sin 2x + 4B \cos 2x + x(-4A \sin 2x - 4B \sin 2x)] + 4x[A \cos 2x + B\sin 2x] 
        = 12 \cos 2x
    \]

    Equating the coefficients we have 
    \begin{align*}
        \begin{cases}
            -4A = 0 &\Rightarrow A = 0\\
        4B = 12 &\Rightarrow B = 3
        \end{cases}
    \end{align*}

    this arrived that $\fbox{$y_p = 3x \sin 2x$}$. The general solution of \eqref{eq:sod2} is 

    \begin{equation*}
        y = c_1 \cos 2x + c_2 \sin 2x + \fbox{$3x \sin 2x$} \label{eq:sod2.1} \tag{{\color{green} $\clubsuit$}}
    \end{equation*}

    From the given IVP conditions, we can determine the constants $c_1$ and $c_2$.
    
    Differentiating \eqref{eq:sod2.1} with respect to $x$,

    \[
        y' = - 2 c_1 \sin 2x + 2 c_2 \cos 2x + 3 (\sin 2x + 2x \cos 2x)
    \]

    and now substitute the initial conditions $y(0) = 3$ and $y'(0) = 4$

    \begin{equation*}
        \begin{cases}
            y(0) = 3 &\Rightarrow c_1 + 0 + 0 = 3\\
            y'(0) = 4 &\Rightarrow 0 + 2c_2 + 0 = 4
        \end{cases}
    \end{equation*}

    On solving, we obtained $c_1 = 3$ and $c_2 = 2$. The general solution for 
    \eqref{eq:sod2} is 
    \[
        y = 3 \cos 2x + (3x + 2) \sin 2x
    \]

\end{solution}

\section{Variation of Parameters}

This method is more powerful than that of undetermined coefficients. It can be used to find 
$y_p$ even for the LDE with variable coefficients:
\begin{equation}
    y'' + p(x)y' + q(x)y = \varphi(x)
\end{equation}

\begin{theorem}[Variation of parameters]
    Consider the differential equation. 

    \begin{equation}
        y'' + p(x)y' + q(x)y = \varphi(x)
    \end{equation}

    Assume that $y_1(t)$ and $y_2(t)$ are a fundamental set of solutions for 
    \begin{equation}
        y'' + p(x)y' + q(x)y = 0
    \end{equation}

    Then a particular solution to the non-homogeneous DE is
    \begin{equation}
        y_p(t) = u'_1 y_1(t) + u'_2 y_2(t)
    \end{equation}
    where 
    \begin{equation}
        u'_1 = -\,\frac{y_2(x) \varphi(x)}{W(y_1, y_2)}, \quad
        u'_2 = \frac{y_1(x) \varphi(x)}{W(y_1, y_2)} 
    \end{equation}

    $W(y_1, y_2)$ is the Wronskian in which defined as 
    \begin{equation}
        W(y_1, y_2) = \begin{vmatrix}
            y_1(x) & y_2(x)\\
            y'_1(x) & y'_2(x)
        \end{vmatrix} = y_1(x)\,y'_2(x) - y_2(x)\,y'_1(x)
    \end{equation}
\end{theorem}

\begin{example}
    Find the general solution of 

    \[
        y'' + y = \sec x, \quad 0 < x < \pi/2
    \]
\end{example}

\begin{example}
    Find the general solution of 

    \[
        y'' -2y' + 2y = e^x \sin x
    \]
\end{example}
\begin{solution}
    This equation is a second-order non-homogeneous DE. The general solution 
    should take the form of 
    
    \[
        y = y_p + y_h
    \]

    We find $y_h$ with characteristic equation 

    \[
        r = \frac{2 \pm \sqrt{(-2)^2 - 4(2)}}{2} = \fbox{$1 \pm i$}
    \]

    has two complex roots, so the general solution of $y_h$ is 

    \[
        y_h = e^{x}[c_1 \cos x + c_2 \sin x]
    \]
    
    Next, we are going to use the method of variation of parameters to determine the 
    particular solution $y_p$. According to the theorem, $y_p$ to the 
    non-homogeneous DE is 
    \[
        y_p(t) = u'_1 y_1(t) + u'_2 y_2(t)
    \]

    Compute the Wronskian,
    \begin{align*}
        W = \begin{vmatrix}
            y_1 & y_2 \\ y'_1 & y'_2
        \end{vmatrix}
        &= \begin{vmatrix}
            e^x \cos x & e^x \sin x \\ -e^x \sin x + \cos x & e^x \sin x + e^x \cos x
        \end{vmatrix}\\
        &= e^x \cos x (e^x \sin x + e^x \cos x) - e^x \sin x (e^x \cos x - e^x \sin x)\\
        &= e^{2x}
    \end{align*}

    and we compute $u'_1$ and $u'_2$,

    \[
        u'_1 = -\, \frac{y_2\, \varphi(x)}{W} = -\, \frac{e^x \sin x (e^x \sin x)}{e^{2x}} = - \sin^2 x
    \]

    \[
        u'_2 = \frac{y_1\, \varphi(x)}{W} = \frac{e^x \cos x (e^x \sin x)}{e^{2x}} = \sin x \cos x
    \]

    Itegrating $u'_1$ and $u'_2$ with respect to $x$,
    \begin{align*}
        u_1 = \int - \sin^2 x \> dx &= \int -\biggl( \frac{1}{2} - \frac{1}{2}\cos 2x \biggr) \> dx \\
        &= -\frac{x}{2} + \frac{\sin (2x)}{4} & \text{(Ignore the constant)}
    \end{align*}

    \begin{align*}
        u_2 = \int \sin x \cos x \> dx &= \frac{1}{2} \int (2 \sin x \cos x) \> dx\\
        &= -\frac{1}{4} \cos 2x + C \\
        &= -\frac{1}{4} + \frac{1}{2} \sin^2 x + C\\
        &= \frac{1}{2} \sin^2 x & \text{(Ignore the constants)}
    \end{align*}

    this yield that the particular solution is 

    \[
        y_p = - \biggl(\frac{x}{2} - \frac{\sin 2x}{4} \biggr) e^x \cos x 
        + \frac{1}{2} \sin^2 x (e^x \sin x)
    \]

    thus the general solution of the non-homogeneous DE is
    
    \[
        y = e^x[c_1 \cos x + c_2 \sin x] - \biggl(\frac{x}{2} - \frac{\sin 2x}{4} \biggr) e^x \cos x 
        + \frac{1}{2} e^x \, \sin^3 x
    \]

    where $c_1$ and $c_2$ are arbitrary constants.
\end{solution}

\section{Variable-Coefficient Equations}

\begin{definition}
    Any linear differential equation (LDE) of the form
    \begin{equation}
        a_n x^n y^{(n)} + a_{n-1} x^{n-1} y^{(n-1)} + \cdots +
    a_{1} x^{1} y^{(1)} + a_{0} y = 0
    \end{equation}
    where the coefficients $a_n, a_{n-1}, \ldots, a_0$ are constants is called an \textbf{Euler equation}.
\end{definition}

\begin{example}
    Solve the IVP:
    \[
        x^2y'' - 2y = 4x - 8, \quad y(1) = 4, \quad y'(1) = -1
    \]
\end{example}
\begin{solution}
    Given 
    \begin{equation}
        x^2y'' - 2y = 4x - 8 \label{eq:sod2.5} \tag{{\color{orange} $\blacktriangle$}}
    \end{equation}

    This is an Euler equation. The substitution $y = x^m$ yields 
    \[
        m(m-1) - 2 = m^2 - m - 2 = (m+1)(m-2) = 0
    \]
    whose roots are $m_1 = -1$ and $m_2 = 2$.

    Hence, 

    \[
        y_h = c_1 y_1 + c_2 y_2 = c_1 x^{-1} + c_2x^2
    \]

    Use the variation of parameters to find $y_p$, a particular solution for \eqref{eq:sod2.5}.

    The standard form of this DE is 

    \[
        y'' - \frac{2}{x^2}y = r(x) = \frac{4x - 8}{x^2}
    \]

    compute Wronskian

    \[
        W = \begin{vmatrix}
            y_1 & y_2\\ y'_1 & y'_2
        \end{vmatrix}
        = \begin{vmatrix}
            x^{-1} & x^2 \\ -x^{-2} & 2x 
        \end{vmatrix} = 3
    \]

\end{solution}

\section{Tutorials}

\begin{mdframed}
    \vspace{-0.25cm}
    \hspace{-0.25cm}
    \begin{Exercise}
        Solve the initial-value problem

        \[
            y'' - 6y' + 8y = 85\cos x, \quad y(0) = 0, \quad y'(0) = 2
        \]
    \end{Exercise}

    \begin{Exercise}
        Find a second order differential equation so that the solution
        \[
            y = C_1 e^{-3x} \cos(4x) + C_2 e^{-3x} \sin(4x) + 4e^{3x}
        \]

        solves the differential equation for any choice of $C_1$ and $C_2$.
    \end{Exercise}

    \begin{Exercise}
        Use the method of variation of parameters to find a particular solution 
        of the given non-homogeneous equation. 
        
        \[
            y'' + y = \sec x \tan x
        \]
        
        Then find the general solution 
        of the equation.
    \end{Exercise}
\end{mdframed}