\chapter{Laplace Transform}

\section{Partial Fraction}

\begin{example}
    Decompose the following fractions as a sum of partial fractions
    \begin{enumerate}
        \item $\displaystyle \frac{2x^2 - x + 4}{x^3 + 4x}$
        \item $\displaystyle \frac{3x^2 -4x + 5}{(x+1)^2(x-2)}$
    \end{enumerate}
\end{example}

\section{Laplace Transform}
    \begin{definition}[Laplace Transform]
        Let $f(t)$ be a function defined for all $t \geq 0$. The Laplace Transform (LT) of 
        $f(t)$ is defined by 

        \begin{equation}
            \mathcal{L}\{ f(t)\} = \int_{0}^{+\infty} e^{-st}\,f(t)\> dt = \lim_{N \to +\infty} \int_{0}^{N} e^{-st}\,f(t)\> dt
        \end{equation}
    \end{definition}

\begin{example}
    Evaluate $\mathcal{L}\{\sin \omega t \}$ and $\mathcal{L}\{\cos \omega t \}$, where 
    $\omega = -\frac{1}{2}+i\frac{\sqrt{3}}{2} = e^{\frac{2}{3}\pi i}$ is a cubic root of unity.
\end{example}
\begin{solution}
    From complex analysis, the De Moivre's theorem state that 

    \[
        \begin{cases}
            e^{i\theta} = \cos \theta + i \sin \theta\\
            e^{-i\theta} = \cos \theta - i \sin \theta\\
        \end{cases}
    \]

    this implies that 
    \[
    \begin{cases}
        e^{i\theta} + e^{-i\theta} = 2\cos \theta\\
        e^{i\theta} - e^{-i\theta} = 2i \sin \theta\\
    \end{cases}
    \]

    We let $\theta = \omega t$, 

    \begin{align*}
        \mathcal{L} \{ \cos \omega t \} &= \frac{1}{2} [ \mathcal{L} \{e^{i\omega t}\} + \mathcal{L} \{e^{-i\omega t}\} ]\\
        &= \frac{1}{2} \biggl[\frac{1}{s - i\omega} + \frac{1}{s - (-i\omega)}\biggr]\\
        &= \frac{1}{2} \biggl[\frac{1}{s - i\omega} + \frac{1}{s + i\omega}\biggr]\\
        &= \frac{1}{2} \frac{(s + i\omega) + (s - i\omega)}{(s - i\omega)(s + i\omega)}\\
        &= \frac{s}{s^2 - (i\omega^2)}\\
        &= \frac{s}{s^2 + \omega^2}
    \end{align*}

\end{solution}[Behavior of $F(s)$]
    
\subsection{Inverse Laplace Transforms}  
Consider $F(s)$ represents the Laplace transform of a function $f(t)$, that is, $F(s) = \mathcal{L}\{f(t) \}$.
we then said that $f(t)$ is the inverse Laplace transform of $F(s)$ and we write

\begin{equation}
    f(t) = \mathcal{L}^{-1} \{F(s)\}
\end{equation}

that is, the original function $f(t)$ itself is the differentiation of the Laplace transform.

\begin{theorem}[Differentiation of Transform]
    Given the transform $F(s)$ to be
    \begin{equation}
        \mathcal{L}\{f(t)\} = F(s) = \int_{0}^{+\infty} e^{-st}\,f(t)\> dt 
    \end{equation}
    then the differentiation of transform is 
    \begin{equation}
        \frac{dF}{ds} = \int_{0}^{+\infty} (-t\,f(t))\, e^{-st} \>dt = \mathcal{L} \{-t\, f(t)\}
    \end{equation}
\end{theorem}

\begin{corollary}
    The inverse transform of $F(s)$ can be write as
    \begin{equation}
        f(t) = - \frac{1}{t} \mathcal{L}^{-1} \{F'(s)\}
    \end{equation}
\end{corollary}

As with Laplace transforms, we've got the following fact to help us take the inverse transform.

\begin{theorem}[Linearity of Inverse Laplace Transforms]
    Given two Laplace transforms $F(s)$ and $G(s)$ then
    \begin{equation}
        \mathcal{L}^{-1} \{aF(s) + bG(s)\} = a\,\mathcal{L}^{-1} \{F(s)\} + b\, \mathcal{L}^{-1} \{G(s)\}
    \end{equation}

    for any constants $a$ and $b$.
\end{theorem}

\begin{example}
    Solve the following IVP problem:
    
    \[
        \begin{cases}
            y'' + 2ty' - 4y = 1\\
            y(0) = y'(0) = 0
        \end{cases}
    \]
\end{example}
\begin{solution}
    Applying Laplace transform to the DE, we have
    \begin{align*}
        &[s^2Y - sy(0) - y'(0)] - 2\biggl(s\frac{dY}{ds} + Y \biggr) - 4Y = \frac{1}{s}\\
        &s^2Y - 2s\frac{dY}{ds} - 6Y = \frac{1}{s}\\
        &-2s \frac{dY}{ds} + (s^2 - 6)Y = \frac{1}{s}\\
        &\frac{dY}{ds} + \frac{6-s^2}{2s} Y = - \frac{1}{2s^2} \label{eq:lpl41} \tag{{\color{red} $\bigtriangleup$}}
    \end{align*}

    upon transform, it now become an 1st-order DE. We further use the integrating factor technique 
    to solve this problem.

    \[
    IF = e^{\int p(s)\> ds} = \exp \biggl\{\int \frac{6 - s^2}{2s}\> ds \biggr\} = s^3\, e^{-s^2/4}
    \]

    multiply IF with \eqref{eq:lpl41},

    \begin{align*}
        \frac{d}{ds} \biggl(Ys^2 e^{-s^2/4}\biggr) &= -\frac{s}{2}\, e^{-s^2/4}\\
        Ys^2 e^{-s^2/4} &= \int -\frac{s}{2}\, e^{-s^2/4} \> ds = e^{-s^2/4} + C & \text{where } C \text{ is a constant}\\
        Y &= \frac{1}{s^3} + \frac{Ce^{s^2/4}}{s^3} \label{eq:lpl41b} \tag{{\color{cyan} $\bullet$}}
    \end{align*}

    As $s \to \infty$, we calculate the limit of \eqref{eq:lpl41b} (use L'Hospital Rule to verify)

    \[
        \lim_{s \to +\infty} Y = \lim_{s \to +\infty} \biggl(\frac{1}{s^3} + \frac{Ce^{s^2/4}}{s^3}\biggr) = 0
    \]

    if and only if $C = 0$. Thus

    \[
        Y = \frac{1}{s^3} \quad \Rightarrow y = \frac{t^2}{2!} = \frac{1}{2}t^2
    \]

\end{solution}

\section{Convolution}

\begin{example}
    Solve the equation 
    \[
        y = t + \int_{0}^{t} y(\tau) \sin(t - \tau) \> d\tau
    \]
\end{example}
\begin{solution}
    The original equation can be rewrite as 
    \[
        y = t + y \star \sin t
    \]

    Applying Laplace Transform on the original equation, we have 
    \[
        Y = \frac{1}{s^2} + Y \cdot \frac{1}{s^2 + 1}
    \]

    Solving for $Y$, 

    \begin{align*}
        Y \biggl(1 - \frac{1}{s^2 + 1}\biggr) = \frac{1}{s^2} &\Rightarrow 
        Y \biggl(\frac{s^2}{s^2 + 1}\biggr) = \frac{1}{s^2}\\
        &\Rightarrow Y = \frac{s^2 + 1}{s^4}\\
        &\Rightarrow Y = \frac{1}{s^2} + \frac{1}{s^4}
    \end{align*}

    Applying inverse Laplace transform on Y, we obtain
    \begin{align*}
        \mathcal{L}^{-1}\{Y\} = \mathcal{L}^{-1} \biggl\{\frac{1}{s^2} + \frac{1}{s^4}\biggr\} &\Rightarrow 
        y = \mathcal{L}^{-1}\biggl\{\frac{1}{s^2}\biggr\} + \mathcal{L}^{-1}\biggl\{\frac{1}{s^4}\biggr\}\\
        &\Rightarrow y = t + \frac{t^3}{3!} = t + \frac{1}{6}t^3
    \end{align*}
\end{solution}

\begin{theorem}[First Shifting Theorem]
    If $\mathcal{L}\{f(t) \} = F(s)$, then $\mathcal{L}\{e^{at} \, f(x)\} = F(s-a)$.
\end{theorem}
\begin{proof}
    \[
        \mathcal{L}\{e^{at} \, f(x)\} = \int_{0}^{+\infty} e^{-st}\, e^{at}\, f(t)\> dt = \int_{0}^{+\infty} e^{-(s-a)t}\, f(t)\> dt = F(s-a)
    \]

    which has been shown.
\end{proof}

The Laplace Second Shifting Theorem, on the other hand, states that the
 Laplace transform of the delayed function
 equals the product of the Laplace transform of the original function and the shifted function.

\section{More Properties of Laplace Transform}

\subsection{Step function}

\begin{definition}[Unit Step Function]
    The step function $H(t - a)$ is defined to be
    \begin{equation}
        H(t - a) = \begin{cases}
            0, & 0 \leq t < a\\
            1, & t \geq a
        \end{cases}
    \end{equation}

    the unit step function is also known as the Heaviside function.
\end{definition}

\begin{theorem}[Second Shifting Theorem]
    Suppose     
\end{theorem}

\subsection{Dirac Delta function}

A unit impulse function is a function that is "on" for a short period and 0 ("off") otherwise

\input{plots/lapl-dirac.tex}

mathematically,

\begin{equation}
    \delta_a(t - t_0) = \begin{cases}
        0, & 0 \leq t_0 < t_0 - a\\
        \frac{1}{2a}, & t_0 - a \leq t_0 < t_0 + a\\
        0, & t_0 \geq t_0 + a
    \end{cases}
\end{equation}

The unit impulse function has the property 

\begin{equation}
    \int_{0}^{+\infty} \delta_a(t - t_0) \> dt = 1
\end{equation}

the limit of this function as $a \to 0$ is the Dirac Delta function. Note that 
$\delta(t - a)$ is not a proper function.

\begin{definition}[Dirac Delta function]
    The Dirac Delta function is defined as 
    \begin{equation}
        \delta(t - a) = \lim_{k \to 0^+} f_k(t) = \begin{cases}
           +\infty, & \text{if } t = a\\
           0, & \text{if } t \neq a
        \end{cases}
    \end{equation}

    where 
    \begin{equation}
        f_k(t) = \begin{cases}
            \displaystyle \frac{1}{k}, & \text{if } a \leq t \leq a+k\\[0.75em]
            0, & \text{otherwise}
        \end{cases}
    \end{equation}
\end{definition}

\begin{example}
    Solve the IVP problem:
    \[
        y'' + 2y' + 5y = 25t - \delta(t - \pi), \quad y(0) = -2, \quad y'(0) = 5
    \]
    where $\delta$ is known as Dirac Delta function.
\end{example}
\begin{solution}
    Applying Laplace transform on the original DE, we have

    \begin{align*}
        [s^2Y - sy(0) - y'(0)] + 2[sY - y(0)] + 5Y &= \frac{25}{s^2} - e^{-\pi s}\\
        s^2 + 2s - 5 + 2[sY + 2] + 5Y = \frac{25}{s^2} - e^{-\pi s}
    \end{align*} 

    Solving for $Y$, 
    \begin{align*}
        &(s^2 + 2s + 5)Y = \frac{25}{s^2} - e^{-\pi s} - 2s + 1\\
        &Y = \frac{25}{s^2(s^2+2s+5)} - \frac{e^{-\pi s}}{s^2 + 2s + 5} + \frac{1-2s}{s^2 + 2s +5}\\
        &Y = \frac{5}{s^2} - \frac{2}{s} - \frac{e^{-\pi s}}{s^2 + 2s + 5}\\
        &Y = \frac{5}{s^2} - \frac{2}{s} - e^{-\pi s}\, F(s) \label{eq:lpl42} \tag{{\color{red} $\diamond$}}
    \end{align*}

    Notice that $F(s)$ can be simplify for ease of inverse transformation,

    \begin{align*}
        F(s) = \frac{1}{s^2 + 2s + 5} &= \frac{1}{(s^2 + 2s + 2) + 2}\\
        &= \frac{1}{(s + 1)^2 + 1}\\
        &= \frac{1}{2} \biggl[\frac{2}{(s + 1)^2 + 2^2}\biggr]\\
    \end{align*}

    inverting $F(s)$ will obtain $f(t) = \displaystyle \frac{1}{2} e^{-t} \sin(2t)$. 

    Furthermore, we again invert \eqref{eq:lpl42},
    \begin{align*}
        y &= 5t - 2 - f(t - \pi)\, H(t - \pi)\\
        &= 5t - 2 - \frac{1}{2}e^{-(t-\pi)} \sin 2(t-\pi)\, H(t - \pi)\\
        &= 5t - 2 - \frac{1}{2}e^{-(t-\pi)} [\sin 2t \cos \pi - \cos 2t \sin 2\pi]\, H(t - \pi)\\
        \Rightarrow y &= 5t - 2 - \frac{1}{2}e^{-(t-\pi)} \sin(2t) \, H(t - \pi)
    \end{align*}

    which this is the solution for the DE.
\end{solution}

\section{Periodic Function}

\begin{definition}[Periodic function]
    The function $f$ is said to be periodic with period $p > 0$, if $f(t + p) = f(t)$ 
    for all $t \in \text{Domain } f$.
\end{definition}

For example, $\sin(t + 2\pi) = \sin(t)$, so the function $\sin(t)$ is a periodic function 
with period of $2\pi$.

\begin{theorem}[Laplace Transform of Periodic Function]
    Suppose $\mathcal{L}\{f(t)\}$ exists and $\exists \hat{\tau} > 0$ s.t. $f(t + \hat{\tau}) = f(t)$ for all 
    $t \geq 0$, then 

    \begin{equation}
        \mathcal{L}\{f(t)\} = \frac{1}{1 - e^{-st}} \int_{0}^{\hat{\tau}} e^{-st}\, f(t)\> dt
    \end{equation}
\end{theorem}

\section{Tutorials}

\begin{mdframed}
    \vspace{-0.25cm}
    \hspace{-0.25cm}
    \begin{Exercise}
        Express the fraction $\displaystyle \frac{x^2 + 7x - 3}{(x-2)(x^2 + 1)}$ as the sum of partial fractions.
    \end{Exercise}

    \begin{Exercise}
        \begin{enumerate}
            \item Show that $\sin 3x = 3 \sin x - 4\sin^3x$.
            \item Hence, using the identity to evaluate $\mathcal{L}\{\sin^3 x\}$.
        \end{enumerate}
    \end{Exercise}

    \begin{Exercise}
        Find $\displaystyle \mathcal{L} \biggl\{\frac{}{}\biggr\}$
    \end{Exercise}

    \begin{Exercise}
        Solve the initial-value problem
        \[
            x'' + 16x = \cos(4t),\quad x(0) = 0,\quad x'(0) =1
        \]
    \end{Exercise}

    \begin{Exercise}
        Solve the equation
        \[
            y' + 4y + 5\int_{0}^{t} y \> dt = e^{-t}, \quad y(0) = 0
        \]
    \end{Exercise}

    \begin{Exercise}
        Solve the integral equation
        \[
            y(t) = e^{-t} - 2\int_{0}^{t} y(u) \cos(t-u) \> du
        \]
    \end{Exercise}

    \begin{Exercise}
        Find the Laplace transform of the periodic function shown in figure below.

        \begin{center}
    \begin{tikzpicture}
    \begin{axis}[
    scale = 0.75,
    height=6cm,
    width=12cm,
    xmin = 0, xmax = 5,
    ymin = 0, ymax = 2,
    axis lines* = left,
    xtick = {0}, ytick = \empty,
    clip = false,
    ]
    % Labels
    \node [right] at (current axis.right of origin) {$t$};
    \node [above] at (current axis.above origin) {$E(t)$};

    \node [below] at (1, 0) {$1$};
    \node [below] at (2, 0) {$2$};
    \node [below] at (3, 0) {$3$};
    \node [below] at (4, 0) {$4$};

    \node [left] at (0, 1) {$1$};

    % Dashed lines (y-axis)
    \addplot[color = red, ultra thick] coordinates {(0, 1) (1, 1)};
    \addplot[color = red, ultra thick] coordinates {(1, 0) (2, 0)};
    \addplot[color = red, ultra thick] coordinates {(2, 1) (3, 1)};
    \addplot[color = red, ultra thick] coordinates {(3, 0) (4, 0)};
    \addplot[color = red, ultra thick] coordinates {(4, 1) (5, 1)};

    % Dashed lines (x-axis)
    \addplot[color = red, dashed] coordinates {(1, 1) (1, 0)};
    \addplot[color = red, dashed] coordinates {(2, 0) (2, 1)};
    \addplot[color = red, dashed] coordinates {(1, 1) (1, 0)};
    \addplot[color = red, dashed] coordinates {(3, 0) (3, 1)};
    \addplot[color = red, dashed] coordinates {(4, 0) (4, 1)};

    \end{axis}
    \end{tikzpicture}
\end{center}
    \end{Exercise}

    \begin{Exercise}
        In the two-mesh network shown below, the switch is closed at $t = 0$ and the voltage source is given by
    $V(t) = 150 \sin(1000t)$. Find the mesh currents $i_1$ and $i_2$.

    \begin{center}
        \begin{circuitikz}[european voltages] \draw
            (0,0) to[sinusoidal voltage source=$V$] (0,4)
              to [short] ++ (1,0)  
              to[cute closing switch] ++ (2,0)
              to[R=10<\ohm>] ++ (2,0)       coordinate (a)
              to[short] ++ (2,0)            coordinate (g)
              to[R=5<\ohm>, *-] ++ (0,-2)   
              to[short] ++ (0,-2)           coordinate (d)
              (a) to[R=5<\ohm>, *-] ++ (0,-2)   coordinate (b)
              (b) to[cute inductor=$0.01H$] ++ (0,-2)    coordinate (c)
              (c) to[short, *-, i=$I_0$] (0,0)  
              (d) to[short, *-] (c);
              \node (a1) [below left=.25cm and .25cm of a] {};
              \node (c2) [above left=.25cm and 3.25cm of c] {};
              \node (c3) [left=.25cm of c2] {};
    
              \node (g1) [below left=.25cm and .25cm of g] {};
              \node (d2) [above left=.10cm and 2.75cm of d] {};
              \path[draw=red, thick,-Triangle] (a1) |- node[red, left, yshift=2cm] {$i_1$} (c2);
              \path[draw=red, thick,-Triangle] (g1) |- node[red, left, yshift=2cm] {$i_2$} (d2);

              \node [above right=1.25cm and .15cm of c] {$+$};
              \node [above right=0.25cm and .15cm of c] {$-$};
        \end{circuitikz}
    \end{center}
    \end{Exercise}
\end{mdframed}