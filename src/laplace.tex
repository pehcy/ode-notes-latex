\chapter{Laplace Transform}

\section{Partial Fraction}

\begin{example}
    Decompose the following fractions as a sum of partial fractions
    \begin{enumerate}
        \item $\displaystyle \frac{2x^2 - x + 4}{x^3 + 4x}$
        \item $\displaystyle \frac{3x^2 -4x + 5}{(x+1)^2(x-2)}$
    \end{enumerate}
\end{example}

\begin{example}
    Evaluate $\mathcal{L}\{\sin \omega t \}$ and $\mathcal{L}\{\cos \omega t \}$, where 
    $\omega = -\frac{1}{2}+i\frac{\sqrt{3}}{2} = e^{\frac{2}{3}\pi i}$ is a cubic root of unity.
\end{example}
\begin{solution}
    From complex analysis, the De Moivre's theorem state that 

    \[
        \begin{cases}
            e^{i\theta} = \cos \theta + i \sin \theta\\
            e^{-i\theta} = \cos \theta - i \sin \theta\\
        \end{cases}
    \]

    this implies that 
    \[
    \begin{cases}
        e^{i\theta} + e^{-i\theta} = 2\cos \theta\\
        e^{i\theta} - e^{-i\theta} = 2i \sin \theta\\
    \end{cases}
    \]
\end{solution}

\section{Convolution}

\begin{example}
    Solve the equation 
    \[
        y = t + \int_{0}^{t} y(\tau) \sin(t - \tau) \> d\tau
    \]
\end{example}
\begin{solution}
    The original equation can be rewrite as 
    \[
        y = t + y \star \sin t
    \]

    Applying Laplace Transform on the original equation, we have 
    \[
        Y = \frac{1}{s^2} + Y \cdot \frac{1}{s^2 + 1}
    \]

    Solving for $Y$, 

    \begin{align*}
        Y \biggl(1 - \frac{1}{s^2 + 1}\biggr) = \frac{1}{s^2} &\Rightarrow 
        Y \biggl(\frac{s^2}{s^2 + 1}\biggr) = \frac{1}{s^2}\\
        &\Rightarrow Y = \frac{s^2 + 1}{s^4}\\
        &\Rightarrow Y = \frac{1}{s^2} + \frac{1}{s^4}
    \end{align*}

    Applying inverse Laplace transform on Y, we obtain
    \begin{align*}
        \mathcal{L}^{-1}\{Y\} = \mathcal{L}^{-1} \biggl\{\frac{1}{s^2} + \frac{1}{s^4}\biggr\} &\Rightarrow 
        y = \mathcal{L}^{-1}\biggl\{\frac{1}{s^2}\biggr\} + \mathcal{L}^{-1}\biggl\{\frac{1}{s^4}\biggr\}\\
        &\Rightarrow y = t + \frac{t^3}{3!} = t + \frac{1}{6}t^3
    \end{align*}
\end{solution}

\begin{theorem}[First Shifting Theorem]
    If $\mathcal{L}\{f(t) \} = F(s)$, then $\mathcal{L}\{e^{at} \, f(x)\} = F(s-a)$.
\end{theorem}

The Laplace Second Shifting Theorem, on the other hand, states that the
 Laplace transform of the delayed function
 equals the product of the Laplace transform of the original function and the shifted function.

\section{More Properties of Laplace Transform}

\subsection{Step function}

\begin{definition}[Unit Step Function]
    The step function $H(t - a)$ is defined to be
    \begin{equation}
        H(t - a) = \begin{cases}
            0, & 0 \leq t < a\\
            1, & t \geq a
        \end{cases}
    \end{equation}

    the unit step function is also known as the Heaviside function.
\end{definition}

\begin{theorem}[Second Shifting Theorem]
    Suppose     
\end{theorem}

\section{Tutorials}

\begin{mdframed}
    \vspace{-0.25cm}
    \hspace{-0.25cm}
    \begin{Exercise}
        Express the fraction $\displaystyle \frac{x^2 + 7x - 3}{(x-2)(x^2 + 1)}$ as the sum of partial fractions.
    \end{Exercise}

    \begin{Exercise}
        \begin{enumerate}
            \item Show that $\sin 3x = 3 \sin x - 4\sin^3x$.
            \item Hence, using the identity to evaluate $\mathcal{L}\{\sin^3 x\}$.
        \end{enumerate}
    \end{Exercise}

    \begin{Exercise}
        
    \end{Exercise}

    \begin{Exercise}
        Solve the initial-value problem
        \[
            x'' + 16x = \cos(4t),\quad x(0) = 0,\quad x'(0) =1
        \]
    \end{Exercise}

    \begin{Exercise}
        Solve the equation
        \[
            y' + 4y + 5\int_{0}^{t} y \> dt = e^{-t}, \quad y(0) = 0
        \]
    \end{Exercise}

    \begin{Exercise}
        Solve the integral equation
        \[
            y(t) = e^{-t} - 2\int_{0}^{t} y(u) \cos(t-u) \> du
        \]
    \end{Exercise}

    \begin{Exercise}
        Find the Laplace transform of the periodic function shown in figure below.

        \begin{center}
    \begin{tikzpicture}
    \begin{axis}[
    scale = 0.75,
    height=6cm,
    width=12cm,
    xmin = 0, xmax = 5,
    ymin = 0, ymax = 2,
    axis lines* = left,
    xtick = {0}, ytick = \empty,
    clip = false,
    ]
    % Labels
    \node [right] at (current axis.right of origin) {$t$};
    \node [above] at (current axis.above origin) {$E(t)$};

    \node [below] at (1, 0) {$1$};
    \node [below] at (2, 0) {$2$};
    \node [below] at (3, 0) {$3$};
    \node [below] at (4, 0) {$4$};

    \node [left] at (0, 1) {$1$};

    % Dashed lines (y-axis)
    \addplot[color = red, ultra thick] coordinates {(0, 1) (1, 1)};
    \addplot[color = red, ultra thick] coordinates {(1, 0) (2, 0)};
    \addplot[color = red, ultra thick] coordinates {(2, 1) (3, 1)};
    \addplot[color = red, ultra thick] coordinates {(3, 0) (4, 0)};
    \addplot[color = red, ultra thick] coordinates {(4, 1) (5, 1)};

    % Dashed lines (x-axis)
    \addplot[color = red, dashed] coordinates {(1, 1) (1, 0)};
    \addplot[color = red, dashed] coordinates {(2, 0) (2, 1)};
    \addplot[color = red, dashed] coordinates {(1, 1) (1, 0)};
    \addplot[color = red, dashed] coordinates {(3, 0) (3, 1)};
    \addplot[color = red, dashed] coordinates {(4, 0) (4, 1)};

    \end{axis}
    \end{tikzpicture}
\end{center}
    \end{Exercise}

    \begin{Exercise}
        In the two-mesh network shown below, the switch is closed at $t = 0$ and the voltage source is given by
    $V(t) = 150 \sin(1000t)$. Find the mesh currents $i_1$ and $i_2$.

    \begin{center}
        \begin{circuitikz}[european voltages] \draw
            (0,0) to[sinusoidal voltage source=$V$] (0,4)
              to [short] ++ (1,0)  
              to[cute closing switch] ++ (2,0)
              to[R=10<\ohm>] ++ (2,0)       coordinate (a)
              to[short] ++ (2,0)            coordinate (g)
              to[R=5<\ohm>, *-] ++ (0,-2)   
              to[short] ++ (0,-2)           coordinate (d)
              (a) to[R=5<\ohm>, *-] ++ (0,-2)   coordinate (b)
              (b) to[cute inductor=$0.01H$] ++ (0,-2)    coordinate (c)
              (c) to[short, *-, i=$I_0$] (0,0)  
              (d) to[short, *-] (c);
              \node (a1) [below left=.25cm and .25cm of a] {};
              \node (c2) [above left=.25cm and 3.25cm of c] {};
              \node (c3) [left=.25cm of c2] {};
    
              \node (g1) [below left=.25cm and .25cm of g] {};
              \node (d2) [above left=.10cm and 2.75cm of d] {};
              \path[draw=red, thick,-Triangle] (a1) |- node[red, left, yshift=2cm] {$i_1$} (c2);
              \path[draw=red, thick,-Triangle] (g1) |- node[red, left, yshift=2cm] {$i_2$} (d2);

              \node [above right=1.25cm and .15cm of c] {$+$};
              \node [above right=0.25cm and .15cm of c] {$-$};
        \end{circuitikz}
    \end{center}
    \end{Exercise}
\end{mdframed}